\section{Exploración Espacial}\label{espacial}

La ruta más expedita para salir de la pobreza es el desarrollo humano. Para impulsarlo debe haber acceso a servicios de salud y educación de buena calidad. Si Colombia quiere tener pros-peridad y justicia social, requiere atender la equidad entre sus zonas rurales y urbanas, entre sus regiones, entre grupos étnicos y entre hombres y mujeres en aspectos como el acceso a la educación, la propiedad de la tierra y la distribución del ingreso.


Según la encuesta de salud sexual y reproductiva de PROFAMILIA del año 2005, la sociedad colombiana ha cambiado radicalmente en los últimos 50 años, con descenso de la tasa global de fecundidad de 6.8 a 2.4, el de mortalidad bruta de 16.7 a 5.5 y el de mortalidad infantil de 123.2 a 25.6 (aún vergonzosa). La esperanza de vida de los colombianos aumentó de 50.6 a 72.2 años, pero también son vergonzosas las diferencias regionales y entre estratos sociales. La mortalidad infantil en el Chocó es tan alta como la africana. Hay 10 puntos de diferencia entre la mortalidad infantil urbana y la rural. El problema de desnutrición infantil continúa sin atención y 12% de los niños son desnutridos crónicos.


Nuevos estudios sugieren que el estrés de ser pobre tiene una peligrosa influencia en la salud. Cuando se comparan los estados socioeconómicos altos y bajos, el riesgo de algunas enfermedades es diez veces mayor. Las personas de estrato socioeconómico bajo tienen dramáticamente más riesgo de enfermar y expectativa de vida más corta.

