\section{Modelos de Regresion}\label{regresion}


En conclusión, vemos los modelos propuestos. Primero sin la poblacion restante como variable independiente, y luego con está. Los resultados se muestran en la Tabla \ref{regresiones} de la página \pageref{regresiones}.



% Table created by stargazer v.5.2.2 by Marek Hlavac, Harvard University. E-mail: hlavac at fas.harvard.edu
% Date and time: jue., jul. 05, 2018 - 11:17:05 a.m.
\begin{table}[!htbp] \centering 
  \caption{Modelos de Regresión} 
  \label{regresiones} 
\begin{tabular}{@{\extracolsep{5pt}}lcc} 
\\[-1.8ex]\hline 
\hline \\[-1.8ex] 
 & \multicolumn{2}{c}{\textit{Dependent variable:}} \\ 
\cline{2-3} 
\\[-1.8ex] & \multicolumn{2}{c}{IDH} \\ 
\\[-1.8ex] & (1) & (2)\\ 
\hline \\[-1.8ex] 
 cabeLog & 0.013$^{***}$ & 0.031$^{***}$ \\ 
  & (0.004) & (0.007) \\ 
  & & \\ 
 restoLog &  & $-$0.030$^{***}$ \\ 
  &  & (0.010) \\ 
  & & \\ 
 Constant & 0.634$^{***}$ & 0.766$^{***}$ \\ 
  & (0.055) & (0.065) \\ 
  & & \\ 
\hline \\[-1.8ex] 
Observations & 32 & 32 \\ 
R$^{2}$ & 0.238 & 0.425 \\ 
Adjusted R$^{2}$ & 0.212 & 0.385 \\ 
Residual Std. Error & 0.037 (df = 30) & 0.033 (df = 29) \\ 
F Statistic & 9.347$^{***}$ (df = 1; 30) & 10.706$^{***}$ (df = 2; 29) \\ 
\hline 
\hline \\[-1.8ex] 
\textit{Note:}  & \multicolumn{2}{r}{$^{*}$p$<$0.1; $^{**}$p$<$0.05; $^{***}$p$<$0.01} \\ 
\end{tabular} 
\end{table} 
Catorce millones de colombianos sobreviven con menos de dos dólares diarios. En total, 64 de cada 100 colombianos están en el umbral de pobreza y algunas regiones padecen niveles casi africanos, como el Chocó.

Según la encuesta de salud sexual y reproductiva de PROFAMILIA del año 2005, la sociedad colombiana ha cambiado radicalmente en los últimos 50 años, con descenso de la tasa global de fecundidad de 6.8 a 2.4, el de mortalidad bruta de 16.7 a 5.5 y el de mortalidad infantil de 123.2 a 25.6 (aún vergonzosa). La esperanza de vida de los colombianos aumentó de 50.6 a 72.2 años, pero también son vergonzosas las diferencias regionales y entre estratos sociales. La mortalidad infantil en el Chocó es tan alta como la africana. Hay 10 puntos de diferencia entre la mortalidad infantil urbana y la rural. El problema de desnutrición infantil continúa sin atención y 12 porciento de los niños son desnutridos crónicos.
\endinput
